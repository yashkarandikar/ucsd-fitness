% THIS IS SIGPROC-SP.TEX - VERSION 3.1
% WORKS WITH V3.2SP OF ACM_PROC_ARTICLE-SP.CLS
% APRIL 2009
%
% It is an example file showing how to use the 'acm_proc_article-sp.cls' V3.2SP
% LaTeX2e document class file for Conference Proceedings submissions.
% ----------------------------------------------------------------------------------------------------------------
% This .tex file (and associated .cls V3.2SP) *DOES NOT* produce:
%       1) The Permission Statement
%       2) The Conference (location) Info information
%       3) The Copyright Line with ACM data
%       4) Page numbering
% 
\documentclass{acm_proc_article-sp}
\usepackage{comment}
\usepackage{url}
\usepackage{float}
\usepackage{needspace}

\begin{document}

\title{Modeling Temporal Evolution in the Endomondo Fitness Data Set}
%
\numberofauthors{1} % section.
%
\author{
\alignauthor
Yashodhan Hemant Karandikar\\
       \email{ykarandi@ucsd.edu}
}

\maketitle
\begin{abstract}

\end{abstract}

\section{Introduction}


\section{Motivation}


\section{Related Work}
\cite{www13} describes various models which account for temporal evolution of user expertise through online reviews on websites such as Amazon, BeerAdvocate, RateBeer, CellarTracker. 

\section{Endomondo Fitness data set}
\label{expAnalysis}
In this section, we introduce the Endomondo fitness data set. \cite{mldataset}. 

\begin{comment}
\begin{figure}[h]
\centering
\includegraphics[scale=0.4]{plots/rating_vs_age}
\caption{\label{plotRatingVsAge}Average rating for different age groups}
\end{figure}

\begin{figure*}
\centering
\includegraphics[scale=0.4]{plots/rating_vs_occupation}
\caption{\label{plotRatingVsOcc} Average rating for various occupations}
\end{figure*}
\end{comment}

\begin{figure*}
\centering
\includegraphics[scale=0.4]{plots/rating_vs_genre}
\caption{\label{plotRatingVsGenre} Average rating for various movie genres}
\end{figure*}

\section{Baseline Model}
\label{secBaselineModel}
This section describes a baseline model to predict the duration of a workout given the distance.

\section{Temporal Evolution of Users}
\label{secTemporalModelUsers}
This section describes a model that attempts to account for temporal evolution of users across workouts. First, we describe the model and then describe the training algorithm.

\subsection{Model}
Given the total distance $d_{ui}$ for the $i$'th workout of user $u$, the predicted duration $T_{ui}'$ of the workout is given by:

$$T_{ui}' = (\alpha_{e_{ui}} + \alpha_{ue_{ui}})(\theta_0 + \theta_1 d_{ui})$$

where $e_{ui}$ is the \emph{experience} level of the user $u$ at the $i$'th workout. Thus, we have one parameter $\alpha_{ue}$ per user $u$ per experience level $e$. Further, we have an intercept term $\alpha_e$ for every experience level $e$, common to all users. The terms $\theta_0$ and $\theta_1$ are global to all users and experience levels. Thus, given $U$ users and $E$ experience levels, we have a total of $UE + E + 2$ parameters. 

\subsection{Training Algorithm}

\section{Temporal Evolution of Workouts}
\label{secTemporalModelWorkouts}

%\needspace{2\baselineskip}
\section{Results}

\begin{table*}
\centering
\begin{tabular}{|c|c|c|c|c|c|c|} \hline
& \# Examples & Variance & Linear Predictor & Collaborative Filtering & Latent Factor & Mahout ALS \\ \hline
Training & 640135 & 0.049970 & 0.029748 {\bf(0.404688)} & 0.023681 {\bf(0.526092)} & 0.020354 {\bf(0.592680)} & {\bf(0.586866)} \\ \hline
Validation & 160033 & 0.049826 & 0.033429 {\bf(0.329085)}  & 0.033488 {\bf(0.327886)} & 0.030545 {\bf(0.386954)} & {\bf(0.392062)} \\ \hline
Test & 200041 & 0.049818 & 0.033776 {\bf(0.322010)} & 0.033779 {\bf(0.321948)} & 0.030765 {\bf(0.382451)} & {\bf(0.388799)}\\ \hline
\end{tabular}
\caption{MSE and $R^2$ obtained using the 3 predictors discussed in this work and Mahout's ALS recommender on the MovieLens dataset. Values in boldface/brackets are $R^2$ values.}
\label{tableResults}
\end{table*}
% end the environment with {table*}, NOTE not {table}!

\section{Conclusion and Future Work}

\bibliographystyle{abbrv}
\bibliography{references.bib}

\end{document}
